%!TEX TS-program = xelatex
\documentclass{friggeri-cv}
\addbibresource{bibliography.bib}

\usepackage{ulem}


\begin{document}
\header{Guilherme}{Nogueira}
       {developer}


% In the aside, each new line forces a line break
\begin{aside}
  \section{contact}
    R. Manuel Henrique Lopes, 220/Apt 16
    05417-050 São Paulo
    Brasil
    ~
    \sout{+55 27 81 23 95 15}
    +33 \phantom{0}7 53 05 52 36
    {\footnotesize skype} guilherme\_nirev
    \href{mailto:guilherme@nirev.org}{guilherme@nirev.org}
    \href{http://nirev.org}{http://nirev.org}
  \section{languages}
    native portuguese
    fluent english
    basic french
  \section{programming}
    {\color{red} $\varheartsuit$} C, Python
    Java, C++, PHP, Android
  \section{skills}
    vraptor, django,
    mysql, postgres,
    CSS \& HTML5,
    jQuery, Linux admin,
    bash scripting
  \section{interests}
distributed systems, multimedia, mobile applications and networks, data structures, data mining, optimization, music, martial arts,
books
\end{aside}


\section{education}

\begin{entrylist}

\entry
  {since $\,\,$2011}
  {Master in Computer Science}
  {University of São Paulo - Brazil}{}
\entry
  {2005–2010}
  {B.Sc. in Computer Science}
  {Universidade Federal do Espírito Santo - Brazil}{}
\end{entrylist}

\section{experience}

\workentry
  {April 2012}{October 2012}
  {Intern at INRIA Paris-Roquencourt}
  {Rocquencourt, Île-de-France - France}
  {Research on supporting non-functional requirements for Wireless Sensor Networks macroprogramming using SunSPOTs.}

\workentry
  {March 2011}{*}
  {Intern for CHOReOS project}
  {University of São Paulo, São Paulo - Brazil}
  {Research and implementation involving: choreography analysis using graph metrics,
  dynamic adaptation techniques, implementation of a testing framework, and cloud computing.}

\workentry
  {August 2010}{February 2011}
  {Analist Developer at RR Sistema}
  {Vitória/ES - Brazil}
  {Porting their implementation of Brazilian DTV middleware to different platforms.
  Mainly programming in C/C++, with low-level and multimedia libraries (GStreamer, DirectFB, pthreads, osal, etc),
  and required knowledge of the Brazilian digital television standards, Linux, embedded systems and
  distributed version control systems.}

\workentry
  {January 2009}{December 2010}
  {Intern at Multimedia and Network Research Lab}
  {Federal University of Espiríto Santo, Vitória/ES - Brazil}
  {Developed an implementation of the Brazilian DTV middleware for Android, and part of it is my B.Sc monograph.}

\workentry
  {August 2008}{December 2008}
  {Intern at Laboratory of Advanced Software SYstems (LASSY)}
  {University of Luxembourg - Luxembourg}
  {Worked in the RESIST project, developing a prototype for an eHealth application using SOA with multiple Web Services,
  using standards such as Apache CXF and WS-Security.}

\workentry
  {January 2008}{March 2008}
  {Intern at CISA Trading S.A.}
  {Trading company, São Paulo/SP - Brazil}
  {Maintaining their internal software solution which manages all of the companies processes.
  The system was written in the Progress4GL language.}

\workentry
  {March 2007}{December 2007}
  {Intern at NINFA laboratory}
  {Federal University of Espiríto Santo, Vitória/ES - Brazil}
  {Joint project with ESCELSA, state's electric power utility, building automated classifiers
  for finding frauds done by its clients. Used optimization heuristics, neural networks and clustering.
  Tools developed in Java, C++ and Matlab.}

\workentry
  {September 2006}{March 2007}
  {Intern at Lettera Soluções}
  {Web developing company, Vitória/ES - Brazil}
  {Developing and managing web sites using CSS, (x)HTML, PHP, ASP, MySQL, PostgreSQL, as well as common Linux admin tools.}


\printbibsection{article}{article in peer-reviewed journal}
\printbibsectionpar{inproceedings}{international peer-reviewed conferences/proceedings}{notkeyword={brazil}}
\printbibsectionpar{inproceedings}{local peer-reviewed conferences/proceedings}{keyword={brazil}}
\printbibsection{misc}{other publications}
\printbibsection{report}{research reports}

\end{document}
